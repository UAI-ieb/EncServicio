\documentclass[]{book}
\usepackage{lmodern}
\usepackage{amssymb,amsmath}
\usepackage{ifxetex,ifluatex}
\usepackage{fixltx2e} % provides \textsubscript
\ifnum 0\ifxetex 1\fi\ifluatex 1\fi=0 % if pdftex
  \usepackage[T1]{fontenc}
  \usepackage[utf8]{inputenc}
\else % if luatex or xelatex
  \ifxetex
    \usepackage{mathspec}
  \else
    \usepackage{fontspec}
  \fi
  \defaultfontfeatures{Ligatures=TeX,Scale=MatchLowercase}
\fi
% use upquote if available, for straight quotes in verbatim environments
\IfFileExists{upquote.sty}{\usepackage{upquote}}{}
% use microtype if available
\IfFileExists{microtype.sty}{%
\usepackage{microtype}
\UseMicrotypeSet[protrusion]{basicmath} % disable protrusion for tt fonts
}{}
\usepackage[margin=1in]{geometry}
\usepackage{hyperref}
\hypersetup{unicode=true,
            pdftitle={Encuesta de Servicio IGS},
            pdfauthor={Unidad de Análisis Institucional},
            pdfborder={0 0 0},
            breaklinks=true}
\urlstyle{same}  % don't use monospace font for urls
\usepackage{natbib}
\bibliographystyle{apalike}
\usepackage{longtable,booktabs}
\usepackage{graphicx,grffile}
\makeatletter
\def\maxwidth{\ifdim\Gin@nat@width>\linewidth\linewidth\else\Gin@nat@width\fi}
\def\maxheight{\ifdim\Gin@nat@height>\textheight\textheight\else\Gin@nat@height\fi}
\makeatother
% Scale images if necessary, so that they will not overflow the page
% margins by default, and it is still possible to overwrite the defaults
% using explicit options in \includegraphics[width, height, ...]{}
\setkeys{Gin}{width=\maxwidth,height=\maxheight,keepaspectratio}
\IfFileExists{parskip.sty}{%
\usepackage{parskip}
}{% else
\setlength{\parindent}{0pt}
\setlength{\parskip}{6pt plus 2pt minus 1pt}
}
\setlength{\emergencystretch}{3em}  % prevent overfull lines
\providecommand{\tightlist}{%
  \setlength{\itemsep}{0pt}\setlength{\parskip}{0pt}}
\setcounter{secnumdepth}{5}
% Redefines (sub)paragraphs to behave more like sections
\ifx\paragraph\undefined\else
\let\oldparagraph\paragraph
\renewcommand{\paragraph}[1]{\oldparagraph{#1}\mbox{}}
\fi
\ifx\subparagraph\undefined\else
\let\oldsubparagraph\subparagraph
\renewcommand{\subparagraph}[1]{\oldsubparagraph{#1}\mbox{}}
\fi

%%% Use protect on footnotes to avoid problems with footnotes in titles
\let\rmarkdownfootnote\footnote%
\def\footnote{\protect\rmarkdownfootnote}

%%% Change title format to be more compact
\usepackage{titling}

% Create subtitle command for use in maketitle
\newcommand{\subtitle}[1]{
  \posttitle{
    \begin{center}\large#1\end{center}
    }
}

\setlength{\droptitle}{-2em}

  \title{Encuesta de Servicio IGS}
    \pretitle{\vspace{\droptitle}\centering\huge}
  \posttitle{\par}
    \author{Unidad de Análisis Institucional}
    \preauthor{\centering\large\emph}
  \postauthor{\par}
      \predate{\centering\large\emph}
  \postdate{\par}
    \date{2019-05-06}

\usepackage{booktabs}
\usepackage{amsthm}
\ifxetex
  \usepackage{polyglossia}
  \setmainlanguage{spanish}
  % Tabla en lugar de cuadro
  \gappto\captionsspanish{\renewcommand{\tablename}{Tabla}  
          \renewcommand{\listtablename}{Índice de tablas}}

\else
  \usepackage[spanish,es-tabla]{babel}
\fi
\makeatletter
\def\thm@space@setup{%
  \thm@preskip=8pt plus 2pt minus 4pt
  \thm@postskip=\thm@preskip
}
\makeatother

\begin{document}
\maketitle

{
\setcounter{tocdepth}{1}
\tableofcontents
}
\chapter*{}\label{section}
\addcontentsline{toc}{chapter}{}

\begin{center}\includegraphics{images/logoieb} \end{center}

\chapter*{Presentación}\label{presentacion}
\addcontentsline{toc}{chapter}{Presentación}

La Encuesta de Servicios, es aplicada desde 2014 de manera periódica y
sistemática a los estudiantes del Instituto Guillermo Subercaseaux. Esta
encuesta, en cuanto su aplicación, tiene una periodicidad semestreal.El
objetivo central es el identificar, desde la percepción de los
estudiantes, el funcionamiento y desempeño de actores, áreas y espacios
a través de los cuales el Instituto entrega su servicio. Para ello, se
cuantifica el grado de satisfacción o insatisfacción de los estudiantes
respecto del uso de dichos actores/servicios/espacios en cada una de las
sedes. De esta manera, los datos permiten determinar cuáles son los
servicios que impactan con mayor o menor intensidad en la satisfacción
integral de los estudiantes.

Este libro ha sido escrito en
\href{http://rmarkdown.rstudio.com}{R-Markdown} empleando el paquete
\href{https://bookdown.org/yihui/bookdown/}{\texttt{bookdown}}

\begin{flushleft}\includegraphics{images/IconoIGS} \end{flushleft}

\chapter{Marco Metodológico}\label{intro}

En este acápite, se detalla la metodología utilizada y que soporta la
validez del estudio y sus resultados. Se especifica el tipo de muestra
con sus parámetros propios y el tipo de instrumento aplicado.

\section{Diseño de la Muestra}\label{requisitos}

En este estudio, el universo válido corresponderá a los estudiantes de
todas las carreras del Instituto, sean vespertinos y diurnos y sean de
educación presencial y/o virtual; además deben poseer los siguientes
atributos:

\begin{itemize}
\tightlist
\item
  Deben tener matrícula vigente hasta la fecha de aplicación.
\item
  Su estado como alumnos debe ser \emph{regular}.
\end{itemize}

Para el cálculo del tamaño de la muestra se ha hecho uso de la fórmula
de muestreo aleatorio simple para proporciones, operando con un 95\% de
confianza, con una varianza -de proporciones- también máxima (p=0,5 y
q=0,5). El error muestral del 4,5\% para cada una de las sedes. Esto da
como resultado una muestra de 1.243 casos a nivel instituto, y un error
muestral a nivel del mismo de un 2,1\%.

\begin{longtable}[]{@{}lcc@{}}
\toprule
Sedes & Matrícula 2018 & Muestra 2018-2\tabularnewline
\midrule
\endhead
Santiago & 1.819 & 541\tabularnewline
Viña del Mar & 258 & 151\tabularnewline
Rancagua & 397 & 188\tabularnewline
Concepción & 299 & 150\tabularnewline
Temuco & 418 & 213\tabularnewline
\textbf{Total} & \textbf{3.191} & \textbf{1.243}\tabularnewline
\bottomrule
\end{longtable}

La base de datos utilizada para el cálculo de las muestras incorpora,
como datos de contacto, el correo electrónico de los alumnos. Mediante
este correo se hará el contacto con ellos(as) para que accedan al
formulario online y así entreguen la información solicitada. Además, y
para asegurar la eficacia de la estrategia, se considera la colaboración
de las direcciones de las sedes del IGS que participen del estudio.

\section{Diseño del Instrumento}\label{diseno-del-instrumento}

El instrumento utilizado para medir la satisfacción del estudiante
considera tanto preguntas cerradas (con opciones de respuestas), además,
y para ampliar el espectro de observación del fenómeno, se añaden en el
instrumento preguntas abiertas que permiten -a los encuestados-
describir mejor su experiencia en relación con los servicios.

En esta oportunidad, la aplicación del instrumento se efectuó mediante
la plataforma web SurveyMonkey. El proceso implica hacer una invitación
a la muestra seleccinada via correo electrónico, el palzo para responder
fue desde el 10 de Noviembre hasta el 23 de octubre de 2018.

\chapter{Resultados}\label{resultados}

A continuación, se muestran los resultados más relevantes de la
encuesta. Existen dos indicadores claves para dar cuenta del grado de
satisfacción/insatisfacción de los estudiantes; la \textbf{Satisfacción
total} y la \textbf{Satisfacción neta}. La primera alude al porcentaje
de alumnos que evalúan a cada servicio, área o espacio con nota 6 ó 7,
en cambios, la satisfacción neta corresponde a la diferencia del
porcentaje de estudiantes que evalúan con nota 6 ó 7, menos aquellos que
evalúan con nota inferior a 5; este valor es expresado en porcentaje. A
nivel tranversal dentro del informe, será este último el valor-indicador
que dará cuenta de la percpeción evaluativa de los estudiantes.

\section{Evaluación Global Instituto Guillermo
Subercaseaux}\label{evaluacion-global-instituto-guillermo-subercaseaux}

\textbf{2.1.1) Evaluación Global del Servicio entregado por IGS}

El siguiente gráfico, señala los resultados histórico del ítem que alude
a la \emph{calidad de los servicios entregados por el Instituto}. Los
datos permiten visualizar una tendencia, en ambas satisfacciones, al
alza. De esta manera, la \emph{satisfacción neta} en la medición actual
(2018-2), alcanza un 51\%, en tanto la \emph{satifacción total} alcanza
un 63\%. Estos valores, al compararse con las mediciones anteriores, son
altos, evidenciándose un retorno a valores registrados en la medición de
2014-2.

\includegraphics{bookdown_intro_files/figure-latex/unnamed-chunk-5-1.pdf}
En el siguiente gráfico, se evalúa sólo el valor de la
\emph{satisfacción neta} a nivel de cada una de las sede del Instituto.
Este dato permite apreciar las fluctuaciones que expresa la satisfacción
neta, una fluctuación cuyos valles y peak son más acentuados en la sede
de \textbf{Concepción} y \textbf{Viña del Mar}. Por el contrario, estos
valores fluctúan con menos intensidad y dentro de un espectro de valores
más altos en la sedes de \textbf{Temuco} y \textbf{Rancagua}. Se puede
apreciar que \textbf{Santiago} manifiesta valores históricamente más
bajos, alcanzando en la medición 2018-2 tan sólo 36\%.

\includegraphics{bookdown_intro_files/figure-latex/unnamed-chunk-7-1.pdf}

\textbf{2.1.2) Percepción de la Calidad de la Atención del IGS}

Otro de los indicadores globales, hace referencia a la \emph{calidad de
la atención que recibe el estudiante}. Esta variable debe ser analizada
en mutua complementación con la variable anterior, puesto que expresa
una alta correlación concepctual, y a su vez, estadística. Es una
variable ``nueva'', en cuanto que sólo ha sido medida en las últimas 3
ocasiones (desde 2017-2). Se puede apreciar, que tal como la variable
anterior, esta variable expresa un alza relevante, específicamente de 22
puntos porcentuales en cuanto a la \emph{satisfacción neta} y en más de
13 puntos en cuanto la \emph{satisfacción total (6-7)}. Por ende, esta
tendencia avala y consolida lo registrado en la variable revisada en el
acápite anterior.

\includegraphics{bookdown_intro_files/figure-latex/unnamed-chunk-8-1.pdf}
Al inspeccionar el comportamiento de la variable comparativamente por
sede, se aprecia un comportamiento de alza-descenso-alza en las sedes de
\emph{Rancagua} y \emph{Viña del Mar}, aunque en la primera de ellas es
más estable en torno a un rango de valores más altos. Se puede apreciar
que las sedes de \emph{Concepción} y \emph{Temuco} registran un repunte
significativo de respecto de la medición anterior, en este mismo aspecto
se debe incluir a \emph{Viña del Mar}. Por último \emph{Santiago},
refleja una tendencia al alza, aunque sus valores son más bajos respecto
de las otras cuatro sedes.

\begin{verbatim}
## Warning: Missing column names filled in: 'X1' [1]
\end{verbatim}

\includegraphics{bookdown_intro_files/figure-latex/unnamed-chunk-10-1.pdf}

\textbf{2.2.3) Disposición a Recomendar el IGS}

La disposición a recomendar un servicio a otro individuo, especialmente
si esta es una persona significativa (familiar o amigo/a), expresa en sí
mismo una experiencia positiva respecto de uso de aquel servicio. Al
observar las cifras en este ámbito, se aprecia que la disposición a
recomendar el Instituto Subercaseaux alcanza un 86\%, sumando aquellos
estudiantes que lo dicen con toda claridad (``Definitivamente = 40\%''),
con aquellos que lo indican como una probabilidad cierta
(``probablemente = 46\%''). Esta cifra expresa un alza de 6 puntos
respecto del dato registrado en la medición anterios (2018-1).

\includegraphics{bookdown_intro_files/figure-latex/unnamed-chunk-12-1.pdf}

El gráfico siguiente muestra el porcentaje de estudiantes que tiene una
disposición positiva frente a la posibilidad de recomendar el Instituto.
Los valores en este dato son más altos, tanto en la actual medición como
en las anteriores, tendencia que se repite en cada una de la sedes. En
este sentido, se constata un desacoplamiento de este dato respecto de
las últimas dos variables generales revisadas; al parecer, el dato y su
tendencia expresa una satisfacción de base donde convergen otros
elementos y aspectos conectados con la trayectoria como estudiantes.

\includegraphics{bookdown_intro_files/figure-latex/unnamed-chunk-14-1.pdf}

\section{Satisfacción con Docentes}\label{satisfaccion-con-docentes}

Dentro de los actores del Instituto, los docentes ocupan un puesto
estratégico dado que son ellos/as quienes interactúan de forma cotidiana
con los estudiantes, una interacción que es fruto de la función
pedagógica propia del Instituto. En este sentido, sus desempeños -en
términos amplio- es parte de ese conjunto de elementos intangible que
los estudiantes tienen presente al momento de generar una opinión. En el
siguiente gráfico, se aprecia la satisfacción neta y total,
registrándose en ambas un incremento de 9 y de 6 puntos,
respectivamente.

\includegraphics{bookdown_intro_files/figure-latex/unnamed-chunk-15-1.pdf}
A nivel de cada una de las sedes, la satisfacción neta de los
estudiantes con los docentes muestra diferencias relevantes. La
satisfacción neta es más alta (en las últimas 3 mediciones), en las
sedes de \emph{Rancagua} y \emph{Temuco}, en ese orden; esto más allá de
que en ambas se registre un descenso en la medición 2018-1. La sede de
\emph{Concepción} tiene un descenso en la medición de 2018-1, pero se
recupera en la medición (2018-2). La sede de \emph{Viña del Mar} es
bastante estable en las tres mediciónes, en torno al 68\% de
satisfacción neta. Por último, en la sede de \emph{Santiago} la
satisfacción neta es baja, y aunque ha venido al alza aún no supera el
umbral del 50\%.

\includegraphics{bookdown_intro_files/figure-latex/unnamed-chunk-17-1.pdf}

Aquellos estudiantes que evalúan con nota inferior a 7 a los docentes,
se les consulta cuáles -desde su perspectiva- son las mejoras que
debiesen ocurrir para lograr un óptimo de satisfacción con respecto al
desempeño docente. En el siguiente gráfico se ha desglosado el dato por
cada sede para acercarnos a fenómenos más localizados. Es poisble
apreciar que en cada una de las sedes lo que se sugiere es mejorar la
``metodología'' de enseñanza, junto con ello aparece como sugerencia el
hecho de mejorar la ``comunicación o claridad expositiva en clases''.

\includegraphics{bookdown_intro_files/figure-latex/unnamed-chunk-19-1.pdf}

\section{Satisfacción con Infraestructura y Otros Espacios Físicos del
IGS}\label{satisfaccion-con-infraestructura-y-otros-espacios-fisicos-del-igs}

\textbf{2.3.1. Satisfacción con la Infraestructura}

En el siguiente gráfico, se aprecia la satisfacción neta y total en las
últimas 3 mediciones. Se aprecia que ha existido un alza sistemática,
aunque leve. Los valores indican que la satisfacción neta está en torno
al 50\% y la satisfacción total en torno al 65\%.

\includegraphics{bookdown_intro_files/figure-latex/unnamed-chunk-20-1.pdf}
El siguiente gráfico permite analizar la satisfacción con la
infraestructura, pero a su vez aquellos espacios que específican de
mejor forma, aunque no completamente, su ámbito. Revisando el gráfico
siguiente, es posible apreciar que el espacio de
\emph{Cafetería/Casino}, corresponde a aquél que genera un menor grado
de satisfacción neta.Por otro lado, ya desde 2017-2, el espacio de
\emph{Biblioteca} registra un alza en la satisfacción neta que alcanza
en la medición actual un 78\%. Otro servicio que ha tenido un repunte
consistente corresponde a \emph{Baños}, alcanzando un 61\% en la última
medición. Partiendo de un piso algo más alto, el espacio de \emph{salas
de clases} también registra un alza sostenida. Por último, laboratorios
es el espacio que manifiesta mayor estabilidad en la diferentes
mediciones.

\includegraphics{bookdown_intro_files/figure-latex/unnamed-chunk-22-1.pdf}
La satisfacción con la infraestructura a nivel de sede se registra en el
siguiente gráfico. Es posible apreciar que \emph{Rancagua} registró en
la medición 2017-2 el valor más alto de satisfacción neta, aunque
después descendió entorno al 60\%. Se les asimila la dinámica que
muestran las sedes de \emph{Temuco} y \emph{Viña del Mar}. La sede de
\emph{Santiago} registra una satisfacción neta históricamente baja, no
superando el 50\%. No obstante, aún más bajo es la satisfacción en la
sede de \emph{Concepción}, donde el valor es incluso negativo (-16\%).
Además, en esta última sede, el valor más alto se registró en 2017-2,
pero sólo alcanzó 11\%, luego bajo al 7\% y llegó en esta última
medición a la cifra antes indicada.

\includegraphics{bookdown_intro_files/figure-latex/unnamed-chunk-24-1.pdf}

Se les preguntó a los estudiantes que evaluaron con nota inferior a 7 a
la infraestructura de la sede, \emph{¿qué mejoras debiesen ocurrir para
lograr un óptimo de satisfacción en este ámbito?}. En infraestructura,
lo que se sugiere es diferente según la realidad de cada sede. Por
ejemplo, en \emph{Rancagua} y \emph{Viña del Mar} se solicita de forma
clara mejorar la \textbf{Cafetería}. En tanto en \emph{Concepción} y
\emph{Temuco} hay una petición clara por mejorar los \textbf{Espacios
comunes}.

\includegraphics{bookdown_intro_files/figure-latex/unnamed-chunk-26-1.pdf}

** 2.3.2. Salas de Clases**

La satisfacción neta relacionada con la \emph{Sala de Clases} registra
en 2018-2 un 61\%, 11 puntos más respecto de la medición anterior
(2017-2). Al observar cómo se comporta este valor por sede, la tendencia
indica que la satisfacción neta en esta ámbito ha aumentado desde una
caída importante en el año 2015, estabilizándose en tres sedes
(\emph{Rancagua}, \emph{Temuco} y \emph{Viña del Mar}). En el caso de
esta última, llegó a registrar valores negativos en los años 2015 y
2016, mejorando de forma significativa los últmos dos años. Por otra
parte, tanto \emph{Concepción}, como \emph{Santiago}, registran valores
comparativamente más bajos en este ámbito.

\includegraphics{bookdown_intro_files/figure-latex/unnamed-chunk-28-1.pdf}
** 2.3.3. Baños**

Según se señaló en el acápite referido a infraestructura-Baños, en
2018-2 se registra un 61\% de satisfacción neta, casi el doble del valor
registrado en 2017-2 (36\%). En las sedes \emph{Viña del Mar} y
\emph{Rancagua} se registraron caídas significativas en los años 2015 y
2016. Lo mismo ocurre en las otras sedes, pero de forma menos acentuada.
Todas las sedes, desde que experimentan una caída tienen un alza más o
menos acentuada, aunque sí constante. En este ámbito, es \emph{Temuco}
la sede con mejor desempeño a nivel histórico y \emph{Santiago} aquella
con los valores más bajos.

\includegraphics{bookdown_intro_files/figure-latex/unnamed-chunk-30-1.pdf}
** 2.3.4. Laboratorios**

En relación con los laboratorios, se aprecia tendencias a la baja hasta
el año 2017-2 en los casos de \emph{Concepción} y \emph{Santiago},
después de dicho año existe un repunte en la medición actual (2018-2).
En el caso de la primera sede este repunte alcanza al 44\%, en cuanto a
la segunda sede, llega al 52\%. En el caso de \emph{Rancagua} y
\emph{Viña del Mar}, estas caídas ocurren en los años 2015-2 y 2016-2,
respectivamente. No obstante, ambas sedes hoy (2018-2) alcanzan valores
de satisfacción neta en torno al 73\%. Es muy relevante poner atención
al descenso sistemático que experimenta la satisfacción neta la sede de
\emph{Temuco}, como es posible ver en el gráfico, estos valores tienen
una tendencia negativa sostenida.

\includegraphics{bookdown_intro_files/figure-latex/unnamed-chunk-32-1.pdf}

** 2.3.5. Cafetería**

En un acápite anterior se observó que el espacio del Cafetería era
aquella que registraba una satisfacción neta más baja. El siguiente
gráfico permite adentrarse en dicha tendencia a nivel de cada una de las
sedes. El gráfico coincide con la sede con mejor desempeño histórico
hasta aquella con peor desempeño. Se puede apreciar que sólo
\emph{Temuco} no registra una tendencia a la baja desde el año 2017-2,
incluso en el caso de \emph{Viña del Mar} se llega a valores negativos
(-26\%). Sólo en el caso de \emph{Temuco} se aprecia una tendencia al
alza débil, registrando un valor menos malo (21\%) en la medición actual
(2018-2).

\includegraphics{bookdown_intro_files/figure-latex/unnamed-chunk-34-1.pdf}

\section{Satisfacción Neta con
Biblioteca}\label{satisfaccion-neta-con-biblioteca}

Anteriormente, se apreció que la \emph{Biblioteca} registra una alta
satisfacción neta en la presente medición (2018-2), en el gráfico
siguiente es posible apreciar cómo es esta tendencia en cada una de las
sedes. Se puede visualizar que existen caídas muy importantes en la
satisfación neta en años puntuales según cada sede, exceptuando la sede
de \emph{Santiago} y \emph{Viña del Mar}. Más allá de estas caídas, ya
desde las mediciones del año 2017 existen incrementos que se han
mantenido en el tiempo, alcanzando en el caso de las sedes de
\emph{Rancagua}, \emph{Santiago} y \emph{Temuco}, valores en torno al
80\%.

\includegraphics{bookdown_intro_files/figure-latex/unnamed-chunk-36-1.pdf}
En relación con la biblioteca, las sugerencias de mejoras de los
estudiantes corresponden principalmente a mejorar su
\textbf{Infraestructura}, esto queda reflejado de forma muy nítida en la
sede de \emph{Concepción} y \emph{Temuco}. A su vez, en la sede de
\emph{Rancagua} la petición se orienta a mejorar los \textbf{Horarios}
de funcionamiento.

\includegraphics{bookdown_intro_files/figure-latex/unnamed-chunk-38-1.pdf}

\section{Satisfacción Neta con Espacios o Servicos
Digitales}\label{satisfaccion-neta-con-espacios-o-servicos-digitales}

Tres son los servicios o espacios que se han clasificado dentro de este
acápite: Plataforma IEB Virtual, Biblioteca Digital (Ebook) y Servicio
Wifi. El siguiente gráfico muestra los resultados obtenidos a nivel
histórico. Comparativamente se aprecia una mayor satisfacción respecto
de plataforma IEB Virtual, con valores que se mueven dentro de un mínimo
de 71\% y un máximo de 81\%. En segundo lugar, se puede apreciar la
valoración del Ebook (Biblioteca Digital), que en un inicio (2015-2)
alcanzaba los 36\%, en cmabio ahora (2018-2) llega al 74\%. El servicio
Wifi genera una baja satisfacción neta, si bien ha remontado desde una
caída brusca en el año 2017-1, no supera en la medición actual el 54\%.

\includegraphics{bookdown_intro_files/figure-latex/unnamed-chunk-40-1.pdf}

** a) Ebook o Biblioteca Digital**

En la medición actual (2018-2), un \textbf{64\%} de los estudiantes
entrevistados indica haber utilizado el servicio de E-Books. A nivel de
cada sede, se visualiza que en \emph{Viña del Mar} y \emph{Concepción}
se regisra una satisfacción menor. Por el contrario, la satisfacción, al
menos en 2018-2, es mayor en la sede de \emph{Rancagua} (83\%).

\includegraphics{bookdown_intro_files/figure-latex/unnamed-chunk-42-1.pdf}

** b) Servicio de Wifi**

EL servicio se Wifi, o sea la posibilidad de los alumnos de acceder a
internet desde su laptop o celular en el espacio de las sedes del
Instituto, refleja una alta valoración en sus usarios en la sede de
\emph{Temuco} (80\%), no obtante en el resto de las sedes esta
satisfacción no supera el 55\%.

\includegraphics{bookdown_intro_files/figure-latex/unnamed-chunk-44-1.pdf}

** c) Plataforma IEB Virtual**

La satisfacción neta del IEB Virtual presentó la mejor evaluación dentro
de los servicios digitales evaluados. A nivel de cada sede, 4 de las 5
sedes alcanzan una satisfacción neta cercana al 80\%, mientras que en
\emph{santiago} la satisfacción neta alcanza un 74\%.

\includegraphics{bookdown_intro_files/figure-latex/unnamed-chunk-46-1.pdf}

\section{Satisfacción con Otros Servicios del
IGS}\label{satisfaccion-con-otros-servicios-del-igs}

En este acápite por \emph{``otros servicios''} se alude a Cajas y
Servicios de Solicitudes. En el primer caso, se ve una tendencia clara
hacia el alza desde el año 2015, llegando en el año 2018 a una
satisfacción neta de 71\%. En cuanto al servicio de solicitudes, su
tendencia es fluctuante, desde un valor inicial (2014-2) de 66\%,
bajando a un 22\% en el año 2016. En el año actual este servicio
registra un 61\% de satisfacción neta.

\includegraphics{bookdown_intro_files/figure-latex/unnamed-chunk-48-1.pdf}
** a) Servicio de Cajas**

Respecto del servicio de Caja por sede, se aprecia que en todas las
sedes hay una caída en el segundo año de medición. Sin embargo, desde
esa caída hay un repunte que se ha mantenido constante hasta la medición
actual.Los repuntes más altos, considerando de referencia la medición
actual, se registran en sede de \emph{Rancagua} y \emph{Temuco}. Más
leves son los repuntes registrados en las sedes de \emph{Concepción} y
\emph{Santiago}.

\includegraphics{bookdown_intro_files/figure-latex/unnamed-chunk-50-1.pdf}

** B) Servicio de Solicitudes**

El servicio de solicitudes tiene un comportamiento fluctuante en
términos de la satisfacción neta. Si se toma como referencia la última
medición (2018-2), se aprecia que todas la sedes mejoran respecto del
2018-1, destacándose la mejora en las sedes de \emph{Rancagua} y
\emph{Temuco}. Le siguen las sedes de \emph{Concepción} y \emph{Viña del
Mar}, por el contrario, se aprecia que la sede \emph{Santiago} un
ascenso leve que sólo alcanza un 44\% de satisfacción neta.

\includegraphics{bookdown_intro_files/figure-latex/unnamed-chunk-52-1.pdf}

\chapter{Análisis Multivariable}\label{analisis-multivariable}

\section{Regresión Logística}\label{regresion-logistica}

Con la finalidad de profundizar en los resultados obtenidos en la
encuesta de servicio 2018-2, se ha formulado dos modelos, uno donde la
variable dependiente corresponda a averiguar qué variables hacen más
probable el hecho de recomendar -o No- el instituto a algún familiar o
amigo; por otro lado, se formula un segundo modelo para detectar qué
variables son significativas en cuanto hacer probable evaluar
positivamente (nota 6 ó 7) el servicio entregado por el Instituto.

\begin{enumerate}
\def\labelenumi{\alph{enumi})}
\tightlist
\item
  Modelo Nº1: Probabilidad de recomendar el Instituto
\end{enumerate}

La disposición a recomendar el Instituto representa, en su base, es una
actitud condicionada por la experiencia que el estudiante ha tenido en
su rol de tal en un ciclo determinado. Por ende, se hace necesario
conocer cuáles, de las variables observadas, son factores que aumentan o
reducen la probabilidad de que el evento ocurra: el evento aquí es el
hecho de recomendar el Instituto.

Para responder al objetivo anterior, se ha hecho uso de la técnica
estadística de regresión logistica multivariable. En el cuadro siguiente
se muestran algunos coeficientes que permiten aludir a la validez del
modelo.

Resumen del Modelo Nº1

\begin{longtable}[]{@{}lcc@{}}
\toprule
Log.de la verosimilitud -2 & R\textsuperscript{2} de Cox y Snell &
R\textsuperscript{2} de Negelkerke\tabularnewline
\midrule
\endhead
342,305 & 0,191 & 0,345\tabularnewline
\bottomrule
\end{longtable}

\begin{itemize}
\item
  \textbf{-2 log de la verosimilitud (-2LL)} mide hasta qué punto un
  modelo se ajusta bien a los datos. El resultado de esta medición
  recibe también el nombre de ``desviación''. El valor es moderado, lo
  ideal es que sea lo más bajo posible, por ello sólo se alude a un
  ajuste sólo aceptable.
\item
  \textbf{La R cuadradro de Cox y Snell} se utiliza para estimar la
  proporción de varianza de la variable dependiente explicada por las
  variables predictoras (independientes). La R cuadrado de Cox y Snell
  se basa en la comparación del log de la verosimilitud (LL) para el
  modelo respecto al log de la verosimilitud (LL) para un modelo de
  línea base. Sus valores oscilan entre 0 y 1. En nuestro caso es un
  valor moderado (0,191) que indica que sólo el 19\% de la variación de
  la variable dependiente es explicada por las variables incluidas en el
  modelo.
\item
  \textbf{La R cuadrado de Nagelkerke} es una versión corregida de la R
  cuadrado de Cox y Snell. La R cuadrado de Cox y Snell tiene un valor
  máximo inferior a 1, incluso para un modelo ``perfecto''. La R
  cuadrado de Nagelkerke corrige la escala del estadístico para cubrir
  el rango completo de 0 a 1. Aquí el valor alude a un modelo aceptable.
\end{itemize}

El siguiente cuadro muestra las variables significativas que son parte
del modelo.

\begin{center}\includegraphics{images/Modelo1} \end{center}

La información del cuadro anterior, permite esbozar lo siguiente:

\begin{itemize}
\item
  Cada una de las variables contenidas en el cuadro han resultado ser
  significativas en términos estadístico, por ende, desempeñan un rol en
  relación a la ocurrencia del evento (recomendar).
\item
  Los valores de la columna b son positivos, lo que indica que cada unas
  de las variables, al incrementarse en una unidad (en este caso de
  notas de 1 a 7), aumentan la probabilidad de ocurrencia del evento. Si
  el signo fuese negativo, indicaría que dicha variable desempeña un
  \emph{rol protector}, osea, evita que ocurra el evento (recomendar el
  IGS).
\item
  El valor de la columna Exp(b) indica cuál de las variable
  seleccionadas tiene un mayor poder para explicar el evento de
  recomendar. En este contexto, este rol lo posee la variable nota a los
  docentes puesto que el valor de Exp(b) es el que más se aleja de 1,
  llegando a 2,314.
\item
  Si bien, como deja en claro el punto anterior, la evaluación de los
  docentes resulta ser clave en el hecho de recomendar o no el
  Instituto, están las otras variables las cuales aluden a espacios
  físicos de uso cotidiano para los alumnos, especialmente el área de
  servicios higiénicos y laboratorios. A su vez, aparece la evaluación
  de servicios de pago o cajas.
\end{itemize}

\begin{enumerate}
\def\labelenumi{\alph{enumi})}
\setcounter{enumi}{1}
\tightlist
\item
  Modelo Nº2: Probabilidad de Evaluar Servicios con Nota 6 o 7
\end{enumerate}

El Instituto define como satisfacción global las respuesta a la pregunta
``\emph{evalúe la calidad de los servicios entregados por el
Instituto}''. En este aspecto, junto al modelo anterior, es necesario
conocer cuál de las variables observadas resulta ser factor en cuanto a
evaluar positiva o negativamente los servicios entregados por el
Instituto, o en otras palabras, que ocurra el evento de calificar con
nota 6 o 7 este tópico.

Al igual que en el modelo Nº1, se ha hecho uso de la técnica estadística
de regresión logistica multivariable. En el cuadro siguiente se muestran
algunos coeficientes que permiten dar cuenta de la calidad estadística
del modelo.

Resumen del Modelo Nº2

\begin{longtable}[]{@{}lcc@{}}
\toprule
Log.de la verosimilitud -2 & R\textsuperscript{2} de Cox y Snell &
R\textsuperscript{2} de Negelkerke\tabularnewline
\midrule
\endhead
494,305 & 0,350 & 0,483\tabularnewline
\bottomrule
\end{longtable}

\begin{itemize}
\item
  \textbf{-2 log de la verosimilitud (-2LL)}. El valor de este
  coeficiente es moderado, lo ideal es que sea lo más bajo posible, por
  ello sólo se alude a una calidad del modelo aceptable.
\item
  \textbf{La R cuadradro de Cox y Snell}. Los valores de este
  coeficiente oscilan entre 0 y 1. En este caso el valor es moderado
  (0,350), lo que indica que el 35\% de la variación de la variable
  dependiente es explicada por las variables incluidas en el modelo.
\item
  \textbf{La R cuadrado de Nagelkerke}. El valor de este coeficiente da
  cuenta de un modelo aceptable.
\end{itemize}

El siguiente cuadro muestra las variables significativas que son parte
del modelo.

\begin{center}\includegraphics{images/Modelo2} \end{center}

La información del cuadro anterior, permite esbozar lo siguiente:

\begin{itemize}
\item
  Cada una de las variables contenidas en el cuadro desempeñan un rol
  estadístico en relación a la ocurrencia del evento (recomendar).
\item
  Los valores de la columna b son positivos, lo que indica que al
  aumentar cada unas de las variables independientes aumenta también la
  probabilidad de ocurrencia del evento (calificar con nota 6 y 7 el
  servicio prestado por el Instituto). Si el signo fuese negativo,
  indicaría que dicha variable desempeña un \emph{rol protector}, osea,
  evita que ocurra el evento.
\item
  El valor de la columna Exp(b) indica cuál de las variable
  seleccionadas tiene un mayor poder para explicar el evento de
  recomendar. En este contexto, al igual como ocurre en el modelo nº1,
  este rol lo posee la variable ``NotaDocentes'', donde el valor de
  Exp(b) es el más alto, llegando a 2,190.
\item
  En este modelo, aparecen relevantes 3 variables relacionadas con el
  aspecto material-espacial del Instituto (Infraestructura, Biblioteca y
  Cafetería), una de espacio virtual (IEB-Virtual). Estas variables
  indican que a medida que aumenta en una unidad la calificación (de
  nota 1 a 7), aumenta la probabilidad que el estudiante evalúe con nota
  6 ó 7 el servicio entregado por el Instituto. Una última variable se
  relaciona con el perfil del estudiante (técnico versus profesional);
  en este caso cuando el estudiante pasa de técnico a profesional
  aumenta la probabilidad de evaluar con nota 6 ó 7 al Instituto en
  cuanto al servico que entrega.
\end{itemize}

\section{Análisis Factorial}\label{analisis-factorial}

Este análisis busca identificar nuevas lógicas subyacentes para
re-estructurar las variables o indicadores de los instrumentos,
conformando así nuevas dimensiones más relevantes para ahondar en la
interpretación de los resultados de las encuestas.

\chapter{Conclusiones}\label{conclusiones}

PENDIENTE

\appendix


\chapter{Anexo 2}\label{rmarkdown}

Anexo 1 \{\#pandoc\}

\bibliography{book.bib,packages.bib}


\end{document}
